%%%%%%%%%%%%%%%%%%%%%%%%%%%%%%%%%%%%%%%%%
% Medium Length Professional CV
% LaTeX Template
% Important note:
% This template requires the resume.cls file to be in the same directory as the
% .tex file. The resume.cls file provides the resume style used for structuring the
% document.
%
%%%%%%%%%%%%%%%%%%%%%%%%%%%%%%%%%%%%%%%%%

%----------------------------------------------------------------------------------------
%	PACKAGES AND OTHER DOCUMENT CONFIGURATIONS
%----------------------------------------------------------------------------------------

\documentclass{resume} % Use the custom resume.cls style
\usepackage{enumitem}
\usepackage[left=1.0in,top=1.0in,right=1.0in,bottom=1.0in]{geometry} % Document margins
%\usepackage[none]{hyphenat} % no hyphenation
\newcommand{\tab}[1]{\hspace{.2667\textwidth}\rlap{#1}}
\newcommand{\itab}[1]{\hspace{0em}\rlap{#1}}
\name{Brennan James Sullivan} % Your name
\address{website: brennansullivan.info \\ brennan.sullivan@bcm.edu} % Your phone number and email
\begin{document}
%----------------------------------------------------------------------------------------
%	EDUCATION SECTION
%----------------------------------------------------------------------------------------
\begin{rSection}{{\bfseries EDUCATION}}
 
    {\bfseries Baylor College of Medicine} \hfill { 2021 - Present }
    \\ Ph.D. in Neuroscience 
    \vspace{0.5\baselineskip}

    {\bfseries Rhodes College} \hfill { 2013 - 2017  }
    \\ B.S. in Neuroscience 
    \end{rSection}


\begin{rSection}{{\bfseries RESEARCH POSITIONS}}
    
    {\bfseries Graduate Student — Laboratory of Jeffrey C. Magee, Ph.D.}
    \\ Howard Hughes Medical Institute \hfill {2022 - Present}
    \\ Department of Neuroscience, Baylor College of Medicine
    \\ Duncan Neurological Research Institute at Texas Children’s Hospital \vspace{0.3\baselineskip}
    \\ \textit{Mechanisms of neocortical learning and memory}

    {\bfseries Research Assistant — Laboratory of Shilpa D. Kadam, Ph.D.} 
    \\ Department of Neurology, Johns Hopkins University School of Medicine \hfill {2017 - 2021}
    \\ Hugo W. Moser Research Institute at Kennedy Krieger Children’s Hospital \vspace{0.3\baselineskip}
    \\ \textit{Development of rationale therapeutic targets for pediatric epilepsy and intellectual disability}

    {\bfseries Undergraduate Research Fellow — Laboratory of Kelly A. Dougherty, Ph.D.} 
    \\ Department of Biology, Rhodes College \hfill {2016 - 2017} \vspace{0.3\baselineskip}
    \\ \textit{ Investigation of A-type K$^+$ channel function and pharmacology}
\end{rSection}

%----------------------------------------------------------------------------------------
%	PUBLICATIONS SECTION
%----------------------------------------------------------------------------------------
\begin{rSection}{{\bfseries PUBLICATIONS}}

    \begin{enumerate}[resume, leftmargin=0pt]
        \item [] Articles [* denotes equal contribution]
         \end{enumerate}

    \begin{enumerate}[resume, leftmargin=2em]
        \item Thomas A.*, Yang W.*, Wang C., Tipparaju S.L., Chen G., {\bfseries Sullivan B.J.}, Swiekatowski K., Tatam M., Gerfen C., Li N. (2023) Superior colliculus cell types bidirectionally modulate choice activity in frontal cortex. \emph{Nature Communications} PMID:
        \item {\bfseries Sullivan B.J.}, Kipnis P.K., Carter B.M., and Kadam S.D. (2021) Targeting ischemia-induced KCC2 hypofunction rescues refractory neonatal seizures and mitigates epileptogenesis in a mouse model. \emph{Science Signaling}. PMID: 34752143
        \item {\bfseries Sullivan B.J.}, Ammanuel S., Kipnis P.K., Araki Y., Huganir R.L., and Kadam S.D. (2020) Low-dose Perampanel rescues cortical gamma dysregulation associated with paravalbumin interneuron GluA2 upregulation in epileptic Syngap1+/- mice. \emph{Biological Psychiatry}. PMID: 32107006
        \item Kipnis P.K.*, {\bfseries Sullivan B.J.*}, Carter B.M., and Kadam S.D. (2020) TrkB agonists prevent post-ischemic BDNF-TrkB mediated emergence of refractory neonatal seizures in CD-1 pups. \emph{JCI Insight}. PMID: 32427585. 
        \item Kang J., Kadam S.D., Elmore J.S., {\bfseries Sullivan B.J.*}, Valentine H.*, Malla A.P., Grace A.A., Rahmim A., Loew L.M., Baumann M., Gjedde A., Boctor E.M., and Wong D.F. (2020) Transcranial photoacoustic imaging of NMDA-evoked focal circuit dynamics in rat forebrain. \emph{Journal of Neural Engineering}. PMID: 32084654. 
        \item Carter B.M., {\bfseries Sullivan B.J.}., Landers J.R., Kadam S.D. (2018) Dose-dependent reversal of KCC2 hypofunction and phenobarbital-resistant neonatal seizures by ANA12. \emph{Scientific Reports}. PMID: 30097625
    \end{enumerate}
    \newpage
    \begin{enumerate}[leftmargin=0pt]
        \item [] Reviews
    \end{enumerate}

    \begin{enumerate}[resume, leftmargin=2em]
        \item {\bfseries Sullivan B.J.} and Kadam S.D. (2021) Brain Derived Neurotrophic Factor in Neonatal Seizures. \emph{Pediatric Neurology}. PMID: 33773288 
        \item Kadam S.D., {\bfseries Sullivan B.J.}, Goayal A., Blue M.E., and Smith-Hicks C. (2020) Rett Syndrome and CDKL5 Deficiency Disorder: From Bench to Clinic. \emph{International Journal of Molecular Sciences}. PMID: 31618813 
        \item Kipnis P.K.,{\bfseries Sullivan B.J.}, and Kadam S.D. (2019) Sex Dependent Signaling Pathways Underlying Seizure Susceptibility and the Role of Chloride Cotransporters. \emph{Cells}. PMID: 31085988
    \end{enumerate}

    \begin{enumerate}[leftmargin=0pt]
        \item [] Chapters
    \end{enumerate}

    \begin{enumerate}[resume, leftmargin=2em]
        \item {\bfseries Sullivan B.J.} and Kadam S.D. (2021) Protocol for Drug Screening with Quantitative Video Electroencephalography in a Translational Model of Refractory Neonatal Seizures. In Experimental and Translational Models to Screen Drugs Effective Against Seizures and Epilepsy. D. Vohora, Ed. Springer, New York
        \item {\bfseries Sullivan B.J.} and Kadam S.D. (2020) The involvement of neuronal chloride transporter deficiencies in epilepsy. In Neuronal Chloride Transporters in Health and Disease. X. Tang, Ed. Elsevier, Amsterdam, Netherlands.
    \end{enumerate}     
\end{rSection}

%----------------------------------------------------------------------------------------
%	INVITED TALKS SECTION
%----------------------------------------------------------------------------------------
\begin{rSection}{{\bfseries INVITED TALKS}}
    \setlist[enumerate,0] {leftmargin=0pt}
    {\bfseries Johns Hopkins University and Kennedy Krieger Institute, Baltimore, MD} \hfill {2022} 
   \\ Pediatric Neuroscience Collaboration Seminar, Virtual Live Content Event
   \\ \emph{Hitting a moving target: KCC2 and pharmacoresistant seizures in the developing brain}.
   \item  {\bfseries 3rd International SYNGAP1 Conference} \hfill {2021}
   \\ Young Investigator Workshop, Virtual Live Content Event
   \\ \emph{An early forecast for the SYNGAP1 storm}.
   \item {\bfseries 14th European Congress on Epileptology, Geneva, Switzerland} \hfill{2020}
    \\  Platform session, Cancelled due to COVID-19
    \\ \emph{The KCC2 functional enhancer CLP290 rescues phenobarbital-resistant neonatal seizures.}
    \item {\bfseries The Scripps Research Institute, Jupiter, Florida} \hfill{2018}
    \\ 2nd International SYNGAP1 Conference
    \\ \emph{Perampanel rescues behavioral state-dependent transitions on qEEG in Syngap1$^{+/-}$ mice.} \vspace{0.3\baselineskip}
    \\ {\emph Selected amongst the young investigators to present at the Main Symposium}.
    \item {\bfseries The Scripps Research Institute, Jupiter, Florida} \hfill{2018} 
    \\ Young Investigator Day at the 2nd International SYNGAP1 Conference 
    \\ {\emph Perampanel rescues behavioral state-dependent transitions on qEEG in Syngap1$^{+/-}$ mice.} 
    \item {\bfseries Rhodes College, Memphis, TN} \hfill{2016}
    \\ Summer Undergraduate Research Symposium
    \\ \emph{A hitchhikers guide to voltage-gated K$^+$  channels and dendritic integration}.
\end{rSection}

%--------------------------------------------------------------------------------
%   POSTER PRESENTATIONS 
%-----------------------------------------------------------------------------------------------
%\begin{rSection}{{\bfseries POSTER PRESENTATIONS}}
 %   \begin{enumerate}[resume, leftmargin=0em]
  %      \item [] {\bfseries Sullivan BJ}, Kipnis PK. Carter BM, and Kadam SD (2019). The functional KCC2 enhancer CLP290 rescues phenobarbital-resistant neonatal seizures in a model of HIE. Annual Meeting of the American Epilepsy Society, Baltimore, MD.
   %     \item [] {\bfseries Sullivan BJ}, Ammanuel S, Kipnis PK, Araki Y, Huganir RL, and Kadam SD (2019). Cortical gamma dysregulation in epileptic Syngap1+/- mice is associated with GluA2 upregulation in PV interneurons. Gordon Research Conference on Inhibition in the CNS: Spatio-Temporal Control of GABAergic Signaling and Its Breakdown in Brain Disorders. Newry, ME.
    %    \item [] {\bfseries Sullivan BJ}, Ammanuel S, Kipnis PK, Araki Y, Huganir RL, and Kadam SD (2018). Interictal spikes and sleep dysfunction in epileptic Syngap1+/- mice progress with age. Annual Meeting of the American Epilepsy Society, New Orleans, LA. {\bfseries Selected for the translational research poster tour.}
     %   \item [] {\bfseries Sullivan BJ}, Ammanuel S, Kipnis PK, Araki Y, Huganir RL, and Kadam SD (2018). Interictal spikes and sleep dysfunction in epileptic Syngap1+/- mice progress with age. Gordon Research Conference on Mechanisms of Epilepsy and Synchronization: Translating Mechanisms into Therapies for Epilepsies. West Dover, VT.
      %  \end{enumerate}
   
       %\end{rSection}

%--------------------------------------------------------------------------------
%    AWARDS SECTION, to become AWARDS and GRANTS 
%-----------------------------------------------------------------------------------------------
\begin{rSection}{{\bfseries AWARDS}}
    NINDS Curing the Epilepsies Conference Travel Award \hfill {2021}
    \\ \emph{CURE Epilepsy} 
   
   14th European Congress on Epileptology Travel Award \hfill{2020}
   \\ \emph{International League Against Epilepsy}
   
   American Epilepsy Society Annual Meeting Travel Award \hfill{2019}
   \\ \emph{International League Against Epilepsy}
   
   Young Investigator Travel Award \hfill{2019}
   \\ \emph{Syngap1 Foundation}
   
   Undergraduate Research Fellowship \hfill{2016}
   \\ \emph{Rhodes College}
       \end{rSection}

%----------------------------------------------------------------------------------------
       %UNIVERSITY SERVICE, and DEI/OUTREACH ACTIVITES 
%----------------------------------------------------------------------------------------
\begin{rSection}{{\bfseries SERVICE \& OUTREACH}}
Ad-Hoc reviewer for \emph{eLife, BRAIN, Molecular Psychiatry, Journal of Neural Engineering, Epilepsy Research, and Neuropharmacology}
\item Skype a Scientist  \hfill {2018 - Present}
\\ \emph{Volunteer Scientist}
\item Baylor College of Medicine Neuroscience Program Bootcamp \hfill {2022 - Present}
\\ \emph{Volunteer Lecturer and Peer Mentor}
\item NINDS-AES Young Investigators Virtual Seminar \hfill {2021} 
\\ \emph{Seminar Co-Organizer, Emerging neurotechnology to disrupt epilepsy}
\item Johns Hopkins Internship in Brain Sciences \hfill{2018 - 2020}
\\ \emph{Mentor}
\item International Leauge Against Epilepsy \hfill {2018 - 2020}
\\ \emph{Young Epilepsy Section Member, Research Task Force}
\item Kids Enjoy Exercise Now  \hfill {2017 - 2020}
\\ \emph{Volunteer Coach}
\item Rhodes College Board of Trustees \hfill {2016 - 2017}
\\ \emph{Student Representative}
\item Rhodes College Student Goverment \hfill {2016 - 2017}
\\ \emph{Chair, Student Goverment Secondhand Smoke Committee}
\item Rhodes College Athletic Department \hfill {2014 - 2017}
\\ \emph{Board Member, Student-Athlete Advisory Committee}
\\ \emph{Student Associate to the Athletic Director}
\end{rSection}

%	NEED TEACHING SECTION HERE

%----------------------------------------------------------------------------------------
% Leadserhip Experience
%----------------------------------------------------------------------------------------
\begin{rSection}{{\bfseries LEADERSHIP EXPERIENCE}}
Rhodes College Varsity Men’s Basketball  \hfill {2013 - 2017}
\\ \emph{Team Captain 2015, 2016, 2017}
\\ \emph{SAA Conference Champion 2015, 2017}
\\ \emph{All-Conference Selection 2015, 2016, 2017}
\\ \emph{Rhodes College Fall Sports Top Men’s Athlete 2016}

\end{rSection}

\end{document}